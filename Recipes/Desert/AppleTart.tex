\thispagestyle{fancy}
\section{Warm Apple Tart With Ice Cream} \label{appletart}
\AddToShipoutPicture*{\AppleTart}

The background image of this apple tart was taken by Samantha Murray. This recipe was originally one of Gordon Ramsay's and modified from there.

\subsection*{Ingredients}
\begin{multicols}{3}
	\begin{itemize}
		\item Puff Pastry
		\item Honey Crisp apples
		\item Cane Sugar
		\item Butter
		\item Powdered Sugar
		\item Favorite Ice Cream
		\item Chocolate Ganache
	\end{itemize}
\end{multicols}

\subsection*{Preparation}

Pre-heat Oven to $190^\circ C$ ($375^\circ$ F). On a large sheet of Parchment paper, roll out puff pastry and cut into desired shape. Gently poke the puff pastry with a fork so that it has holes in various places (this keeps it from bubbling up while cooking). Thinly slice cored and peeled apples and lay atop the pastry overlapping each other so that the pastry is covered. Brush top with melted butter and sprinkle cane sugar over top pastry to cover apples. Place in oven to caramelize and until apples are tender and pastry is golden (about 25-30 minutes). Once finished, lightly sift powdered sugar over top and caramelize with blow-torch and serve Warm with a scoop of Vanilla or your favorite ice cream. For extra finesse, make a chocolate ganache to drizzle over top.

\subsection*{Tips}

Puff Pastry can be bought in store but it is hard to find a good puff pastry using good ingredients. If you are bold you can make your own, but I buy mine from Whole Foods which uses great ingredients and works well with this recipe. When baking this, it is best to use a pan with edges on it in case the caramelized sugar runs off of the edge. My favorite combination to use with this is a black cherry ice cream with a dark chocolate ganache found in section \ref{ganache}.