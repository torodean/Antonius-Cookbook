\thispagestyle{fancy}
\section{Spicy Mexican Soup}
\AddToShipoutPicture*{\MexicanSoup}
This recipe was adapted from the youtube video ``Spicy Mexican Soup with Tortillas \& Salsa - Gordon Ramsay.'' I would highly recommend watching 
\subsection*{Ingredients}
\begin{multicols}{3}
	\begin{itemize}
		\item Red Onions
		\item Habanero (or chipotle)
		\item 1 Teaspoon Cumin seeds
		\item 1 Teaspoon Oregano
		\item 2 Clove Garlic
		\item Olive Oil
		\item 1 Tablespoon Brown Sugar
		\item 1 Tablespoon Tomato Puree
		\item 1 Tomato
		\item 1 Can of Kidney Beans
		\item 1 Orange Bell Pepper
		\item $\frac{1}{2}$-1 cup vegetable stock
		\item Cilantro or Coriander
	\end{itemize}
\end{multicols}

\subsection*{Preparation}

Begin by drizzling olive oil in a hot pot. Add finely sliced red onions, bell pepper and habanero. Add finely diced garlic, oregano and cumin seeds and reduce down until ingredients begin to brown. Drizzle olive oil again generously which will help reduce spice. Add brown sugar to coat ingredients in the light caramel. Add tomato puree and blend well. Add one whole diced tomato, kidney beans, and vegetable stock. Let simmer and stir occasionally. The longer you cook the dish the hotter (spicier) it will become. When done, add fresh cilantro or Coriander and mix in. 

This dish is well served with avocado and cheese or with a side of garlic bread. Garlic bread can be made simply by buttering your favorite bread, adding garlic powder and sea salt, and then baking until golden brown and crispy. 

\subsection*{Tips}

Habanero works well with this dish if you can handle a good deal of spice. If not, you may want to use something less potent like chipotle or chili peppers. You can also leave the spice out entirely but the flavor from the spice adds significantly to this dish.