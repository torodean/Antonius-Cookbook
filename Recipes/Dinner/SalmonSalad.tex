\thispagestyle{fancy}
\section{Blackened Salmon on a Bed of Fresh Mixed Greens}
\AddToShipoutPicture*{\SalmonSalad}

\subsection*{Ingredients}
\begin{multicols}{3}
	\begin{itemize}
		\item Salmon Fillet
		\item Rosemary
		\item Basil
		\item Thyme
		\item Sea Salt \& Pepper
		\item Mixed Greens
		\item Cress
		\item Mint
		\item Lemon
		\item 1 Apple
		\item Raisins
		\item Olive Oil
		\item Hemp Seeds (optional) 
	\end{itemize}
\end{multicols}

\subsection*{Preparation}

Begin by Scoring (slicing) the skin of the salmon parallel to the length of the salmon fillet with about a third of an inch in between each score so that each slice has a depth about halfway through the salmon. Please basil, thyme and rosemary (or a combination of them) within the slices followed by salt \& pepper. Gently drizzle olive oil over the skin side of the salmon fillet. In a hot pan, drizzle olive oil and place salmon skin side down. Cook on a medium heat until fillet are about a third done then flip and cook the top for about 15 seconds by tilting fillet so that it sits in any oils left in pan.

While cooking salmon, gently dice mixed greens and place on plate topped with cress leaves, diced apple pieces and raisins. Serve salmon once cooked on top of mixed greens and garnish with lemon slices, hemp seeds, and mint leaves. This dish is served well with a nice olive oil and lemon dressing (see below).

\textbf{Dressing: } In a dressing bottle, mix 2 parts olive oil, 2 parts water, 1 part lemon juice, and then lightly add salt and pepper. You can also add other seasonings like Basil and Thyme if you're feeling up to it. Shake well and lightly drizzle over salmon dish.

\subsection*{Tips}
If the Salmon Fillets are not already scaled, scale them before scoring. This is an adaptable dish depending on the mixed greens you use. Some mixed greens will go well with different items other than apples and grapes. I use gourmet lettuces and a hint of kale when making this dish.
