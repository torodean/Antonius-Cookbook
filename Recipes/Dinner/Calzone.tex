\thispagestyle{fancy}
\section{Calzone}
\AddToShipoutPicture*{\Calzone}
If you don't have time to make your own, you can generally pick up a pizza dough from your local pizzeria!
\subsection*{Ingredients}
\begin{multicols}{3}
	\begin{itemize}
		\item Pizza Dough
		\item Pasta or Pizza Sauce
		\item Mozzarella Cheese
		\item Flour
		\item Salt \& Pepper
	\end{itemize}
\end{multicols}

\subsection*{Ground Beef \& Pepper}
This combination is perfect for using up leftover ground beef made while using tacos.
\begin{multicols}{3}
	\begin{itemize}
		\item Ground Beef
		\item Mushrooms
		\item Red Bell Pepper
		\item Red Onion
		\item Basil
		\item Thyme
		\item Butter
		\item Garlic Powder
	\end{itemize}
\end{multicols}

\subsection*{Preparation}

Preheat oven to $215^\circ$C (425$^\circ$F). Begin by Flouring hands, a rolling pin, and the dough lightly (this will prevent sticking). Roll out dough (or toss if you're adventurous) into circle. Place pizza sauce on half of the dough followed by Mozzarella cheese. Then place other favorite ingredients. Roll empty side of pizza dough on top of the side with ingredients and seasonings and use thumb to make a nice seal between the top and bottom layer. Take a sharp knife (lightly floured) and score the top of the calzone. Mix in a bowl melted butter, garlic powder, and sea salt and brush over top of the calzone. Place on a baking tray and bake in oven for 18-20 minutes or until golden brown.

\subsection*{Tips}

When making pizza or calzones, it is best to wash all of the ingredients well before hand and allow them to dry completely before using. This can prevent the crust from getting soggy.