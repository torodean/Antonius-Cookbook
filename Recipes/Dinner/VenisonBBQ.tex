\thispagestyle{fancy}
\section{Venison with Homemade Barbecue Sauce}
\AddToShipoutPicture*{\VenisonBBQ}

\subsection*{Ingredients}
\begin{multicols}{3}
	\begin{itemize}
		\item Small Tomato's
		\item Mushrooms
		\item Venison
		\item Olive Oil
		\item Salt \& Pepper
		\item 2 Tablespoons Butter
		\item Garlic
	\end{itemize}
\end{multicols}

\subsection*{Barbeque Sauce}
This Barbeque Sauce Recipe is derived from the YouTube video "Smoky Pork Sliders with BBQ Sauce - Gordon Ramsay." I would highly recommend watching it to see how it is cooked.
\begin{multicols}{3}
	\begin{itemize}
		\item 1 Onion
		\item 3 Cloves Garlic
		\item 2 Teaspoons Olive Oil
		\item 1 Tablespoon Brown Sugar
		\item 1 Teaspoon Smoked Paprika
		\item 1 Tablespoon Ketchup
		\item 2 Tablespoons Apple Cider Vinegar
	\end{itemize}
\end{multicols}

\subsection*{Preparation}
\textbf{Barbecue Sauce:} In a hot pan, drizzle olive oil and add finely diced garlic and onions. Saut\'{e} until browning then add brown sugar and caramelize. Add smoked paprika, mix well and then add vinegar. Reduce mixture to remove sourness from vinegar. Once reduced fully, add ketchup and cook until desired thickness is reached.

\vspace{0.25cm}

\textbf{Venison:} Preheat oven to $205^\circ$C (400$^\circ$F). Season Venison generously with salt \& Pepper. In a hot pan, drizzle olive oil. Seer venison on all sides then melt butter over top. Add peeled Garlic cloves to pan and set venison onto garlic so it is not touching the pan (This will help it cook evenly in the oven). Place in oven and let cook for 12-15 minutes\footnote{Time to cook heavily depends on size and shape of meat.} or to desired wellness. In a small pan on low heat, drizzle olive oil and add tomatoes and mushrooms (bulb side down). Season tomatoes and mushrooms and let cook until soft and wrinkled. After venison is cooked, baste with butter and remove from pan. Let venison rest for 10 minutes before cutting (It will continue to cook slightly during this process).

\subsection*{Tips}

In the original recipe, Worcestershire sauce is added with the ketchup to add extra flavor. When cooking tomatoes slowly like we are in this recipe, it is helpful to buy tomatoes still on the vine that way we can set them into the pan with the stem and remove them the same way. It also holds them together which makes for easy plating. It is a good idea to always poke meat when cooking with two fingers. Depending on the meat and firmness of it when touching, this can give you a very good idea of how cooked it is (This comes with practice).