\thispagestyle{fancy}
\section{Quinoa Patties with Tomato Relish}
\AddToShipoutPicture*{\QuinoaPatties}

This recipe was introduced to me by Mike Roosa.

\subsection*{Ingredients}
\begin{multicols}{3}
	\begin{itemize}
		\item 1 Cup Quinoa
		\item 4 Eggs
		\item $\frac{1}{3}$ Cup Shredded Cheese
		\item 3 Spring Onions
		\item 3 Cloves Garlic
		\item $\frac{1}{2}$ Teaspoon Sea Salt
		\item 1 Cup Steamed Kale
		\item 1 Cup Flour
		\item Olive Oil
		\item Buns or bread
		\item Tomato Relish (optional)
	\end{itemize}
\end{multicols}


\subsection*{Preparation}

Rinse quinoa thoroughly and place in saucepan with 2 cups of water. Allow grains to soak for 15 minutes then bring to boil and reduce to a simmer while covered with a lid. Cook until the quinoa has absorbed all of the liquid and is tender. Let quinoa cool to room temperature. Beat eggs in a separate dish, then mix together all ingredients except olive oil and bread in large bowl. The mixture should be moist but not runny. Form patties or appropriate serving sizes and prepare to cook them. On a large skillet, drizzle olive oil and place patties onto skillet. Cover and let cook for about 8 minutes on each side or until both sides appear cooked and center is not raw.

\textbf{Tomato Relish: } This recipe goes well with a tomato relish topping but can be served as one likes. A good tomato relish can be found in section \ref{SteakSandwich} but is optional. If you choose not to make these with a relish, you can simply serve them on bread or topped with lettuces, parsley, and tomato slices for a nice finish!

\subsection*{Tips}
This recipe is best when using cheeses like Parmesan, Asiago, and Romano. Instead of spring onions, yellow or white onions can be used instead if finely chopped. The more patties that are placed on the skillet at a time can affect the cooking times and they will all be drawing on the heat of the pan. For this reason, do not overcrowd the pan while cooking. 